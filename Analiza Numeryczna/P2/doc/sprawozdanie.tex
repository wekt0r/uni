\documentclass[10pt,wide]{mwart}
\usepackage{amsfonts}
\usepackage{amssymb}
\usepackage{graphicx}
\usepackage{svg}
\usepackage[utf8]{inputenc}
\usepackage{tikz}
\usepackage{tabularx}
%\usepackage{polski}
\usepackage[centertags]{amsmath}
\usepackage{amsthm}
\usepackage{newlfont}
\usepackage[polish]{babel}
\usepackage[T1]{fontenc}
\usepackage{listingsutf8}
\usepackage{xcolor}
\usepackage{url}
\newtheorem{lemat}{Lemat}
\newtheorem{tw}{Twierdzenie}
\newtheorem{przyklad}{Przykład}
\newtheorem{wn}{Wniosek}
\newtheorem{zad}{Zadanie}
\theoremstyle{definition}
\newtheorem{df}{Definicja}
\newtheorem{fc}{Fakt}
\newtheorem{uw}{Uwaga}
\newtheorem{szk}{Szkic dowodu}
\renewcommand*{\figurename}{Wykres}
\addto\captionspolish{\renewcommand{\figurename}{Wykres}}



%\textwidth 16cm
%\textheight 23.5cm
%\topmargin -1cm
%\oddsidemargin 0.5cm
%\evensidemargin 0.5cm
%\def\thefootnote{\arabic{footnote})}


\pagestyle{plain}
\begin{document}
\title{\textbf{Pracownia nr 2 z Analizy Numerycznej}\\
Sprawozdanie do zadania \textbf{P2.13}}
\author{Wiktor Garbarek}
\date{Wrocław, Listopad 2017}

\maketitle
 \thispagestyle{empty}
 \section{O interpolacji klasycznej i Hermite'a.}
 Interpolacja jest bardzo ważnym zagadnieniem w kontekście pomiarów czy upraszczania obliczeń. W wielu dziedzinach nauki, gdy wykonujemy jakieś pomiary,
 to niestety możemy to zrobić tylko i wyłącznie na zbiorze dyskretnym. Często jednak znając wyniki pomiarów czy też obliczeń w dwóch punktach
 (gdzie zrobiliśmy to relatywnie prosto) chcielibyśmy wiedzieć co się dzieje w przedziale pomiędzy (szczególnie, jeśli przewidujemy, że zjawisko jest ciągłe).
 Oprócz tego, jeśli przybliżamy to co się dzieje w danym przedziale, chcielibyśmy wiedzieć jak bardzo się to przybliżenie \'mogło pomylić\' oraz, rzecz jasna, chcielibyśmy,
 by ten błąd był jak najmniejszy.

 \section{O problemach związanych z klasyczną interpolacją i ich rozwiązaniach.}

 \section{Interpolacja Hermite'a - strona teoretyczna.}
Gdy rozpatrujemy klasyczną interpolację, to istnienie wielomianu stopnia \(n\) przechodzącego przez \(n+1\) różnych punktów (zwanych też węzłami) jest w pewien sposób intuicyjnie oczywiste - wygląda jak rozwiązanie układu \(n \) równań ze względu na \(n\) zmiennych. Gdy chcemy być nieco bardziej formalni, to wystarczy zwrócić uwagę, że pewna macierz w tym układzie jest macierzą Vandermonde'a i jej wyznacznik jest niezerowy. Sprawa robi się mniej intuicyjna dla interpolacji Hermite'a - czy na pewno możemy dostać wielomian, który nie tylko w pewnych ustalonych punktach osiąga ustalone wartości, ale też pochodne w tych punktach osiągają ustalone wartości - i to nie tylko pierwsze pochodne, ale też drugie czy dalsze!
\begin{tw} (Jednoznaczność istnienia wielomianu Hermite'a)
Niech dane będą: liczba naturalna \(k\), liczby \(m_0, m_1, ..., m_k \in \mathbb{N}_+\), parami różne węzły \(x_0, x_1, ..., x_k\) oraz wielkości rzeczywiste \(y_i^{(i)}\) (\(i = 0,1,...,k; j = 0,1,...,m_k - 1\)). Przyjmijmy \(n := m_0 + m_1 + ... + m_k - 1 \). Wtedy istnieje dokładnie jeden wielomian \(H_n\) stopnia co najwyżej \(n\) spełniający następujące warunki:
\begin{equation}
H_n^{(j)}(x_i) = y_i^{(j)}
\end{equation}
dla \(i = 0, 1, ..., k\) oraz \(j = 0, 1, ..., m_k - 1\).

\begin{proof}
Okazuje się, że pomysł konstrukcji wielomianu Hermite'a oparty na postaci Newton'a klasycznego wielomianu interpolacyjnego, po pewnych ulepszeniach jest matematycznie poprawny, tj. spełnione są równania (1).
Oznaczmy \(s_i := \sum_{j=0}^{i-1} m_i\), gdzie \(s_0 = 0\). Zauważmy, że każda liczba \(0 \leq l \leq n\) ma jednoznaczne przedstawienie w postaci \(l = s_i + j\), gdzie \(0 \leq i \leq k\) oraz \(0 \leq j \leq m_i - 1\).
Zdefiniujmy zatem wielomiany węzłowe następująco:
\begin{equation}
p_{s_i + j}(x) = (x-x_i)^j \cdot \Pi_{j=0}^{i-1}(x - x_j)^{m_j}
\end{equation}
Wtedy
\begin{equation}
H_n(x) = \sum_{l=0}^{n}b_lp_l(x) = \sum_{i=0}^{k}\sum_{j = 0}^{m_{i-1}} b_{s_i + j} p_{s_i + j}(x)
\end{equation}
gdzie dla \(l = s_i + j\) to
\begin{equation}
b_l = \frac{y_i^j - \Big(\sum_{q=0}^{l-1}b_q \cdot p_q(x)\Big)^{(j)}(x_i)}{j!\cdot p_{s_i}(x_i)}
\end{equation}
TODO.
\end{proof}
\end{tw}

 \section{Algorytm konstrukcji wielomianu Hermite'a.}

\section{Porównanie interpolacji Hermite'a do interpolacji klasycznej}

\section{Wnioski o użyteczności interpolacji }


\begin{thebibliography}{9}
\itemsep10pt
\bibitem{JMJ} J. i M. Jankowscy, \emph{Przegląd metod i algorytmów numerycznych}, cz. 1, WNT, 1981.
\bibitem{MS} Michelle Schatzman, \emph{Numerical analysis: a mathematical introduction}, Clarendon Press, Oxford, 2002
\bibitem{CK} W. Cheney, D. Kincaid, \emph{Analiza numeryczna}, WNT, 2006.
\bibitem{approx} \url{https://en.wikipedia.org/wiki/Approximations_of_pi} (Z dnia 10.11.2017).
\bibitem{history} \url{https://en.wikipedia.org/wiki/Chronology_of_computation_of_pi} (Z dnia 10.11.2017).
\end{thebibliography}
\end{document}
